\documentclass[a4paper]{article}

% Encoodaus, joka sopii suomenkielellä (esim. ä ja ö)
\usepackage[utf8]{inputenc}
\usepackage[T1]{fontenc}

% Suomenkielinen tavutus
\usepackage[finnish]{babel}

% Viitteet
\usepackage{natbib}

% Otsikkojen päätteetön fontti
\usepackage{sectsty}
\allsectionsfont{\sffamily\large}

% Viitteiden merkit
\bibpunct{(}{)}{;}{a}{,}{,}

\begin{document}

\title{\small T-76.5612 Software Project Management \\ Visiting Lecture 4 summary \\ \huge F-Secure - Scrum-simulaatio}
\date{13.2.2012}
\author{Mikko Koski \\ mikko.koski@aalto.fi \\ 66467F}
\maketitle

\normalsize

% What were the most important things that you learned during the game? 
% Did the game change your comprehension about Scrum? 
% How do you feel this kind of simulation worked as a training tool (good/bad/improvement suggestions)? 
% Feel free to include verbal feedback to the F-Secure trainers in the end of your text. We'll collect your feedback and deliver it to the trainers anonymously.

\section{Luennon sisältö}

Tämän kerran luennon sisältö oli hyvin poikkeava perinteisestä luennosta. Pelasimme kolmen tiimin voimin Scrum-prosessia simuloivaa peliä. Simulaation veti tiimi F-Securelta. F-Securen työntekijät toimivat pelissä Scrum-mastereina sekä tuotteen omistajina (engl. product owner)

\section{Mielipiteitä simulaatiosta}

Mielestäni tämänkaltainen simulaatio toimii todella hyvänä opetuskeinona. Scrum-prosessin teoriatiedon pystyy toki oppimaan teoriaa lukemalla, mutta käytännön soveltamisen opettamisen kannalta simulaatio oli mielestäni todella onnistunut.

\section{Mitä opin simulaatiosta}

Yhtenä suunnittelupokerin parhaita puolia on se, että pokerin keinoin jokainen tiimin jäsen saa tasavertaisen oikeuden ilmaista oman näkemyksensä kunkin työtehtävän suuruudesta. Keskustelun avulla pyritään löytämään konsensus. Tärkeintä on kuitenkin, että ryhmän hiljaisimmatkin pääsevät ilmaisemaan mielipiteensä.

Simulaation aikana emme käyttäneet aikataulusyistä suunnittelupokeria. Tämä vaikutti selkeästi ryhmämme arvioihin. Arvioinnin aikana oli selkeästi havaittavissa, että muutamat äänekkäimmät hallitsivat keskustelua ja pystyivät näin myös vaikuttamaan arviointiin todella voimakkaasti. Hiljaisimmat eivät välttämättä sanoneet tehtävän koosta mielipidettään ollenkaan. Todellisessa projektissa tämä olisi mielestäni todella huono tilanne. Simulaation aikana tämän edellämainitun ongelman huomattuani vakuutuin siitä, että suunnittelupokeria kannattaa käyttää aina kun mahdollista.

Arvioinnin apuna käytimme hieman erilaista arviointiskaalaa kuin mitä käytimme aikaisemmalla luennolla, jossa harjoittelimme suunnittelupokeria. Aikaisemmalla luennolla valitsimme tehtävistä pienimmän ja annoimme sille yhden pisteen. Tämän jälkeen kaikki suuremmat tehtävät olivat kooltaan joko saman kokoisia tai vähintään tuplasti suurempia. Mutta mitä jos työtehtävä onkin vain puolitoista kertaa suurempi?

Tähän ongelmaan käytimme Scrum-simulaatiossa seuraavaa ratkaisua: valitsimme pienimmän tehtävän ja annoimme sille pistemäärän kolme. Tällä keinoin pystyimme antamaan vain vähän isommalle tehtävällä esimerkiksi pistemäärän neljä.



Scrum vaikuttaa teoriassa prosessina todella hyvältä, mutta käytännössä Scrum-tiimit joutuvat kohtaamaan yllättäviä ongelmia. Simulaatiossa tuotteen omistajana toiminut F-Securen työntekijä kuvasi hyvin näitä ongelmia: asiakas ei välttämättä aina tiedo mitä haluaa, asiakas ei aina osaa kertoa kaikkia vastauksia, jos toimittaja ei osaa kysyä oikeita kysymyksiä. Scrumin ideana on, että päivämäärät eivät jousta, mutta tarvittaessa projektin laajuutta voidaan säätää. Asiakkaat eivät useinkaan ymmärrä tätä ja tästä aiheutuu ongelmia Scrum-tiimille. Esimerkiksi simulaatiossa tuotteen omistaja oli kiinnostunut siitä, saadaanko kaikki tehtävät tehtyä määräaikaan mennessä, muttei kovin kiinnostunut vastaamaan mikä tehtävistä on tärkein siltä varalta, että kaikkea ei saadakaan tehtyä.

\section{Palautetta simulaation järjestäjille}

Opetussimulaation onnistuminen riippuu todella paljon järjestäjien osaamisesta. Jos järjestäjät ovat perehtyneet simulaatioon ja harjoitelleet sen pitämistä, on simulaatiolla suuri potentiaali onnistua.

Tällä kertaa saimme nauttia todella osaavista simulaation vetäjistä. Ainakin oman ryhmämme Scrum-master oli todella hyvin perillä omasta tehtävästään.

Erityisen ison kiitoksen ansaitsee kuitenkin tuotteemme omistaja. Todellisessa elämässä asiakas ei todellakaan toimi optimaalisesti, joten mielestäni oli hienoa, että simulaatiossakin tuotteen omistaja toimi epätäydellisesti. Tämä opetti myös ryhmälle tärkeän opin ongelmista, joita Scrum-tiimi joutuu kohtaamaan.

\end{document}