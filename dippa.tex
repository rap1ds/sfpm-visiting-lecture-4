\documentclass[a4paper]{article}

% Encoodaus, joka sopii suomenkielellä (esim. ä ja ö)
\usepackage[utf8]{inputenc}
\usepackage[T1]{fontenc}

% Suomenkielinen tavutus
\usepackage[finnish]{babel}

% Viitteet
\usepackage{natbib}

% Otsikkojen päätteetön fontti
\usepackage{sectsty}
\allsectionsfont{\sffamily\large}

% Viitteiden merkit
\bibpunct{(}{)}{;}{a}{,}{,}

\begin{document}

\title{\small T-76.5612 Software Project Management \\ Visiting Lecture 4 summary \\ \huge F-Secure - Scrum-simulaatio}
\date{13.2.2012}
\author{Mikko Koski \\ mikko.koski@aalto.fi \\ 66467F}
\maketitle

\normalsize

\section{Luennon sisältö}

Tämän kerran luennon sisältö oli hyvin poikkeava perinteisestä luennosta. Pelasimme kolmen tiimin voimin Scrum-prosessia simuloivaa peliä. Simulaation veti tiimi F-Securelta. F-Securen työntekijät toimivat pelissä Scrum-mastereina sekä tuotteen omistajina (engl. product owner)

\section{Mielipiteitä simulaatiosta}

Mielestäni tämänkaltainen simulaatio toimii todella hyvänä opetuskeinona. Scrum-prosessin teoriatiedon pystyy toki oppimaan teoriaa lukemalla, mutta käytännön soveltamisen opettamisen kannalta simulaatio oli mielestäni todella onnistunut.



\section{Mitä opin?}

\end{document}